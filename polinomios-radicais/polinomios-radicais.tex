\documentclass{article}
\usepackage[utf8]{inputenc}
\usepackage[brazil]{babel}
\usepackage{amsmath}
\usepackage{amsfonts}
\usepackage{amssymb}
\usepackage{amsthm}
\usepackage{cancel}

\title{Polinômios Radicais}
\author{}
\date{}

\newtheorem{theorem}{Teorema}

\begin{document}

\maketitle

\section{Introdução}

Primeiramente vamos definir o conceito de polinômios radicais de forma clara, já que é conceito que eu mesmo criei e nomeiei por tanto defini de acordo com minha curiosidade matemática.

Definimos polinômios radicais como uma equação que pode escrita como: 

\[
\mathcal{P}(x) = \sum_{i=1}^{n} a_i x^{1/i} + a_0 = 0
\]

Onde os expoentes $1/i$ pertencem ao conjunto das frações unitárias $1/2, 1/3, \dots, 1/n$, e $a_i$ são os coeficientes. 

Para o caso específico de um polinômio Radical de grau 3 temos:

\[
\mathcal{P}(x) = a\sqrt[3]{x} + b\sqrt{x} + cx + d = 0.
\]

\section{Equação do Polinômio Radical do Segundo Grau}

Para resolvermos uma equação quadratica Radical:

\[
a\sqrt{x} + bx + c = 0.
\]

Temos a seguinte fórmula

\[
x = \frac{a}{b}\left(\frac{a \pm \sqrt{\Delta_{\text{rad}}}}{2b}\right)- \frac{c}{b}
\]

Onde $\Delta_{\text{rad}}= a^2-4bc$

\clearpage

\section{Prova}

Intuição: Para resolvermos uma equação quadrática Radical devemos convertê-la para uma equação quadrática comum.

Dado

\[
a\sqrt{x} + bx + c = 0.
\]

Isolamos a raiz com menor potência, logo:

\[
\sqrt{x}= \frac{-bx - c}{a}.
\]

Potenciamos os dois lados da equação por 2:

\[
x= \left( \frac{-bx - c}{a}\right)^2
\]

expandimos:

\[
x= \frac{b^2x^2+ 2bcx+c^2}{a^2}
\]

\[
\leadsto
a^2 x= b^2x^2+ 2bcx+c^2
\]

\[
\leadsto
b^2x^2+ 2bcx+c^2 - a^2x = 0
\]

E assim isolamos temos a equação quadrática:

\[
b^2x^2+ (2bc-a^2)x+c^2 = 0
\]

Agora iremos resolver utilizando a fórmula das equações quadráticas.

\subsection{Aplicando Equação do Segundo Grau}

\[
x = \frac{-(2bc - a^2) \pm \sqrt{(2bc - a^2)^2 - 4b^2c^2}}{2b^2}.
\]

\subsection{Passo 1: Simplificar o discriminante $\Delta$}
O discriminante é dado por:

\[
\Delta = (2bc - a^2)^2 - 4b^2c^2.
\]

Expandindo o termo $(2bc - a^2)^2$:
\[
(2bc - a^2)^2 = 4b^2c^2 - 4a^2bc + a^4.
\]

Agora subtraímos $4b^2c^2$:
\[
\Delta = (4b^2c^2 - 4a^2bc + a^4) - 4b^2c^2.
\]

Cancelando $4b^2c^2$:
\[
\Delta = -4a^2bc + a^4.
\]

Separamos o delta em duas partes destacando $a^2$:

\[
\Delta = a^2(-4bc + a^2).
\]

\subsection{Passo 2: Substituir o discriminante na fórmula}
Agora fica como:

\[
x = \frac{-2bc + a^2 \pm \sqrt{a^2(a^2-4bc)}}{2b^2}.
\]

Colocaremos $a^2$ para fora da raiz quadrada como $\pm a$ (repare que $\pm (\pm a) = \pm a$, sendo que estado de um $\pm$ não afeta diretamente o outro $\pm$).

\[
x = \frac{-2bc + a^2 \pm a\sqrt{a^2-4bc}}{2b^2}.
\]

Podemos destacar o $- \frac{c}{b}$ da seguinte forma:

\[
x = \frac{a^2 \pm a\sqrt{a^2-4bc}}{2b^2}- \frac{c}{b}
\]

Definiremos nossa $\Delta_{\text{rad}}= a^2-4bc$, logo a fórmula para resolução os coeficientes de Polinômio Radical do Segundo Grau é a seguinte:

\[
x = \frac{a^2 \pm a\sqrt{\Delta_{\text{rad}}}}{2b^2}- \frac{c}{b}.
\]

ou

\[
x = \frac{a}{b}\left(\frac{a \pm \sqrt{\Delta_{\text{rad}}}}{2b}\right)- \frac{c}{b}
\]

\section{Teorema dos Polinômios Radicais (Para equações de Grau 2)}

\begin{theorem}
Para toda equação no formato:

\[
a\sqrt{x} + bx + c = 0.
\]

Definimos as seguintes constantes:

\[
\alpha = b \\
\beta = -a \\
\gamma = c
\]

Existe o seguinte mapeamento linear:

\[
x=\frac{\beta}{\alpha}y-\frac{\gamma}{\alpha}
\]

De forma que satisfaz a seguinte equação quadrática:

\[
\alpha y^2 + \beta y + \gamma = 0 
\]
\end{theorem}

\begin{proof}
O $x$ da equação do Polinômio Radical do Segundo Grau é definida por:

\[
x = \frac{a}{b}\left(\frac{a \pm \sqrt{\Delta_{\text{rad}}}}{2b}\right)- \frac{c}{b}
\]

Como $\Delta_{\text{rad}}= a^2-4bc$ vamos expandir para:

\[
x = \frac{a}{b}\left(\frac{a \pm \sqrt{a^2-4bc}}{2b}\right)- \frac{c}{b}
\]

Faremos as substituições que envolvem $\alpha = -b$ , $\beta = a$ e $\gamma = c$, logo:

\[
x = -\frac{\beta}{\alpha}\left(-\frac{-\beta \pm \sqrt{(-\beta)^2-4\alpha \gamma}}{2\alpha}\right)- \frac{\gamma}{\alpha}
\]

\[
\leadsto 
x = (-1)\frac{\beta}{\alpha}\left((-1)\frac{-\beta \pm \sqrt{(-\beta)^2-4\alpha \gamma}}{2\alpha}\right)- \frac{\gamma}{\alpha}
\]

\[
\leadsto 
x = \cancel{(-1)(-1)}\frac{\beta}{\alpha}\left(\frac{-\beta \pm \sqrt{(-\beta)^2-4\alpha \gamma}}{2\alpha}\right)- \frac{\gamma}{\alpha}
\]

Sabemos que $(-\beta)^2=\beta^2$ para $\beta \in \mathbb{Z}$, logo:

\[
x = \frac{\beta}{\alpha}\left(\frac{-\beta \pm \sqrt{\beta^2-4\alpha \gamma}}{2\alpha}\right)- \frac{\gamma}{\alpha}
\]

Reparemos que dado $\alpha y^2 + \beta y + \gamma = 0$, se aplicarmos a equação do segundo grau:

\[
y = \frac{-\beta \pm \sqrt{\beta^2-4\alpha \gamma}}{2\alpha}
\]

Dado o seguinte fato acima faremos a seguinte substituição na fórmula de $x$:

\[
x = \frac{\beta}{\alpha}y- \frac{\gamma}{\alpha}
\]

Com isso provamos a existência do mapeamento linear acima.
\end{proof}

\section{Teorema Generalizado dos Polinômios Radicais}

Para fazer : )

\end{document}