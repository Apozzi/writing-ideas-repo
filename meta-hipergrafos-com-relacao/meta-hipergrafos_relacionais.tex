\documentclass{article}
\usepackage[portuguese]{babel}
\usepackage{amsmath}
\usepackage{amsthm}
\usepackage{amsfonts}
\usepackage{unicode-math}
\usepackage{geometry}
\geometry{a4paper, margin=1in}

\newtheorem{theorem}{Teorema}

\title{Meta-Hipergrafos com Relações}
\author{} 

\begin{document}

\maketitle

\section{Introdução}

Estive pensando nessa ideia há algum tempo. Comecei com ``Hipergrafos com Dependências'', mas logo percebi que havia relações mais interessantes e mais generalizáveis, como implicação e implicação com negação. Quis criar condições mais interessantes que dependessem do caminho escolhido, inventei a ideia de Meta-Hipervértices (que é um conjunto de vértices). A ideia foi se expandindo e se tornou algo extremamente generalizável.
\\
% Em relação nomenclatura de conceitos ``Meta-Hipergrafos com Relações'' é o mais adequado, agora caso necessite ser mais especifico ``Grafo do Pozzi``.

\section{Definições}

Um Meta-Hipergrafo com Relações é definido por \( U = (V, H, E_V, R_V, M) \), onde:

\begin{itemize}
    \item \( V \) é um conjunto finito de elementos chamados vértices (ou nós).
    \item \( H \) é um conjunto finito de conjuntos que é uma partição das vértices chamada de Meta-Hipervértices.
    \item \( E_V \) é um conjunto de Meta-Hiperarestas direcionadas. Cada Meta-Hiperaresta \( e \in E_V \) é um par ordenado \( e = (A_e, B_e) \), com \( A_e, B_e \subseteq V \), \( A_e \neq \emptyset \) e \( B_e \neq \emptyset \).
    \item \( R_V \), chamada de Meta-Hiperarestas de relação, é um conjunto de Meta-Hiperarestas direcionadas. Cada Meta-Hiperaresta de relação \( d \in R_V \) é um triplo ordenado \( d = (A_d, B_d, R_d) \), com \( A_d, B_d \subseteq V \), \( A_d \neq \emptyset \) e \( B_d \neq \emptyset \), e \( R_d \in \{\implies, \not\!\!\!\implies\} \).
    \item \( M: V \to \mathbb{N} \cup \{\infty\} \), onde \( M(v) \) é o número máximo de visitas permitidas ao vértice \( v \) em um caminho.
\end{itemize}

Um Meta-Hipercaminho \( P_H \) de um Meta-Hipervértice \( u \) para um Meta-Hipervértice \( v \) em \( U \) é uma sequência de Meta-Hipervértices em \( H \):

\[ P_H = (w_0, w_1, \dots, w_k) \]

onde:

\begin{itemize}
    \item \( w_0 = u \) e \( w_k = v \) (com \( k \geq 1 \)) e \( w_i \in H \),
    \item \( \forall i \in \{1, \dots, k\}, \exists e \in E_V \) tal que \( e = (A_e, B_e) \wedge A_e \subseteq w_{i-1} \wedge B_e \subseteq w_i \).
\end{itemize}

O conjunto de Meta-Hipervértices no Meta-Hipercaminho \( P_H \) é \( MHVert(P_H) = \{ w_0, w_1, \dots, w_k \} \).

Um Caminho \( P_V \) induzido por \( P_H \) é:

\[ P_V = (v_0, \dots, v_k) \]

onde:

\begin{itemize}
    \item \( v_i \in w_i \) para todo \( i \),
    \item \( \forall i = 1, \dots, k: \exists (A_e, B_e) \in E_V \) tal que \( v_{i-1} \in A_e \subseteq w_{i-1} \), \( v_i \in B_e \subseteq w_i \).
\end{itemize}

O conjunto de vértices no caminho \( P_V \) é \( Vert(P_V) = \{ v_0, v_1, \dots, v_k \} \).
\newpage

Um Meta-Hipercaminho \( P_H = (w_0, \dots, w_k) \) é válido em Meta-Hipergrafo com Relações se, além de satisfazer as condições de Meta-Hiperarestas em \( E_V \), todos os seus prefixos consecutivos \( P_H' = (w_0, \dots, w_j) \) (para cada \( 1 \leq j \leq k \)) satisfazem, definindo \( P_H'' = (w_0, \dots, w_{j-1}) \) se \( j \geq 2 \) (e \( P_H'' \) vazio se \( j = 1 \), com \( \bigcup_{w \in MHVert(P_H'')} w = \emptyset \))  as seguintes restrições:

\begin{itemize}
    \item \( \forall (A_d, B_d, \implies) \in R_V: (B_d \cap \bigcup_{w \in MHVert(P_H')} w \neq \emptyset) \to (A_d \cap \bigcup_{w \in MHVert(P_H')} w \neq \emptyset) \), e se \( (B_d \cap \bigcup_{w \in MHVert(P_H')} w \neq \emptyset) \wedge (B_d \cap \bigcup_{w \in MHVert(P_H'')} w = \emptyset) \), então \( (A_d \cap \bigcup_{w \in MHVert(P_H'')} w \neq \emptyset) \).
    \item \( \forall (A_d, B_d, \not\!\!\!\implies) \in R_V: (A_d \cap \bigcup_{w \in MHVert(P_H')} w \neq \emptyset) \to (B_d \cap \bigcup_{w \in MHVert(P_H')} w = \emptyset) \).
    \item \( \forall v \in V: |\{ w \in MHVert(P_H') \mid v \in w \}| \leq M(v) \).
\end{itemize}

Um caminho \( P_V = (v_0, \dots, v_k) \) é válido se todos os seus prefixos consecutivos \( P_V' = (v_0, \dots, v_j) \) (para cada \( 1 \leq j \leq k \)) satisfazem, definindo \( P_V'' = (v_0, \dots, v_{j-1}) \) se \( j \geq 2 \) (e \( P_V'' \) vazio se \( j = 1 \), com \( \{v_i \mid i \in \emptyset \} = \emptyset \)) as seguintes restrições:

\begin{itemize}
    \item \( \forall (A_d, B_d, \implies) \in R_V: (B_d \cap \{v_0, \dots, v_j\} \neq \emptyset) \to (A_d \cap \{v_0, \dots, v_j\} \neq \emptyset) \), e se \( (B_d \cap \{v_0, \dots, v_j\} \neq \emptyset) \wedge (B_d \cap \{v_0, \dots, v_{j-1}\} = \emptyset) \), então \( (A_d \cap \{v_0, \dots, v_{j-1}\} \neq \emptyset) \).
    \item \( \forall (A_d, B_d, \not\!\!\!\implies) \in R_V: (A_d \cap \{v_0, \dots, v_j\} \neq \emptyset) \to (B_d \cap \{v_0, \dots, v_j\} = \emptyset) \).
    \item \( \forall v \in V: |\{ i \mid 0 \leq i \leq j, v_i = v \}| \leq M(v) \).
\end{itemize}

\section{Teoremas}

\begin{theorem}[Contradição por Dependência Cíclica]
\hfill
\\
\\
Seja \( S_R = \bigcup_{(A_d, B_d, \implies) \in R_V} \{ A_d, B_d \} \).
\\
\\
Seja \( G_{+} = (S_R, E_R) \) o grafo dirigido com \( E_R = \{ (A_d, B_d) \mid (A_d, B_d, \implies) \in R_V \} \).
\\

\textbf{Se \( G_{+} \) contém um ciclo dirigido, então não existe \( P_H \) válido nem \( P_V \) válido em \( U \) que comece de um Meta-Hipervértice \( u \) ou vértice \( v_0 \) tal que os vértices ativados no início não pertençam aos conjuntos do ciclo (i.e., \( u \cap \left( \bigcup_{S_i \in C} S_i \right) = \emptyset \) para \( P_H \), ou \( v_0 \notin \bigcup_{S_i \in C} S_i \) para \( P_V \), onde \( C \) é o ciclo).}
\end{theorem}

\begin{proof}
Por absurdo. Foco em \( P_H \) (análoga para \( P_V \)).

Suponha \( P_H = (w_0, \dots, w_k) \) válido com \( w_0 \cap \left( \bigcup_{S_i \in C} S_i \right) = \emptyset \), e suponha que o caminho ativa algum \( S_l \in C \) em algum \( t_{S_l} > 0 \).

Como o caminho entra no ciclo de fora, seja \( S_l \) o primeiro conjunto do ciclo ativado, com \( t_{S_l} = \min \{ t_S \mid S \in C \} \).

Pela estrutura do ciclo, existe \( S_m \to S_l \) (pois todo vértice em ciclo tem dependência), então pela condição estrita para \( (S_m, S_l, \implies) \), em \( j = t_{S_l} \), \( S_m \cap \bigcup_{l=0}^{j-1} w_l \neq \emptyset \).

Mas \( t_{S_m} < t_{S_l} \), contradizendo a minimalidade de \( t_{S_l} \) (pois \( S_m \in C \)).

Propagando para trás no ciclo, a entrada de fora requer uma ativação prévia dentro do ciclo, impossível sem violar a minimalidade ou a estrita precedência.

Assim, nenhum caminho de fora pode entrar no ciclo sem contradição, implicando ausência de tais \( P_H \) válidos.
\end{proof}

\newpage
\begin{theorem}[Troca de Relações Inversas em Caminhos Paralelos]
\hfill

Seja \( U = (V, H, E_V, R, M) \) um Meta-Hipergrafo com relações, onde:

\begin{itemize}
    \item Existem vértices iniciais \( i \in V \), finais \( f \in V \), e \( v_r \in V \) tal que caminhos de \( i \) para \( v_r \) passam obrigatoriamente por \( f \).
    \item Existem dois caminhos paralelos de \( i \) para \( f \): um via \( v_1 \in V \) (i.e., arestas conectando \( i \to v_1 \to f \)), outro via \( v_2 \in V \) (i.e., \( i \to v_2 \to f \)), sem arestas cruzadas ou alternativas.
    \item \( R = \{ (A_1, A_r, \implies) \} \), onde \( A_1 \subseteq V \) contém \( v_1 \) mas não \( v_2 \), e \( A_r \subseteq V \) contém \( v_r \).
    \item \( M(v) = 1 \) para todo \( v \in V \) (proibindo repetições).
    \item Arestas adicionais \( f \to v_r \).
\end{itemize}

Seja \( U' = (V, H, E_V, R', M) \), com \( R' = \{ (A_2, A_r, \not\!\!\!\implies) \} \), onde \( A_2 \subseteq V \) contém \( v_2 \) mas não \( v_1 \).
\\

\textbf{Os conjuntos de Meta-Hipercaminhos válidos \( P_H \) e caminhos válidos \( P_V \) de Meta-Hipervértices contendo \( i \) para Meta-Hipervértices contendo \( v_r \) em \( U \) coincidem com os de \( U' \).}
\end{theorem}

\begin{proof}
Os possíveis Meta-Hipercaminhos candidatos de \( \{i\} \) para \( \{v_r\} \) são sequências passando por \( \{v_1\} \) ou \( \{v_2\} \), depois \( \{f\} \), e \( \{v_r\} \) (outros violam \( E_V \) ou \( M \)).

Em \( U \): Para caminhos via \( v_1 \), prefixos ativando \( A_r \) (i.e., \( v_r \)) já ativam \( A_1 \) (via \( v_1 \)), satisfazendo \( \implies \). Para via \( v_2 \), ativa \( A_r \) sem \( A_1 \), violando \( \implies \).

Em \( U' \): Para via \( v_1 \), \( A_2 \) não ativado, satisfazendo \( \not\!\!\!\implies \) (premissa falsa). Para via \( v_2 \), ativa \( A_2 \) e \( A_r \), violando \( \not\!\!\!\implies \).

Logo, apenas caminhos via \( v_1 \) são válidos em ambos. Análogo para \( P_V \).
\end{proof}

\newpage

\begin{theorem}[Meta-HiperGrafos com Relações Isomórficos a Meta-Hipergrafos sem Relações]
Um Meta-Hipergrafo com relações \( U = (V, H, E_H, R, M) \) é tal que cada Meta-Hipervértice em \( H \) contém apenas um elemento (singleton), correspondente ao seu vértice respectivo em \( V \) (i.e., \( H = \{ \{v\} \mid v \in V \} \)), simulando um grafo direcionado padrão com relações lógicas sobre ativações de vértices.

Suponha que exista uma relação \( (A_i, A_j, \implies) \in R \), com \( A_i = \{v_i\} \), \( A_j = \{v_j\} \), tal que \( v_i \) é necessária para qualquer caminho válido que contenha \( v_j \) ou ative vértices além de \( v_j \) em subgrafos dependentes.

Suponha também que existam caminhos paralelos de um vértice inicial \( s \in V \) para um vértice convergente \( t \in V \) (fechamento de caminhos): um ramo passando por \( v_i \) (permitindo \( v_j \) e além), outro por \( v_x \in V \) (sem \( v_i \), e portanto incapaz de ativar \( v_j \) ou além devido à relação).

Os caminhos podem ser separados em com \( v_i \) (válidos para além de \( v_j \)) e sem \( v_i \) (limitados, não alcançando além de \( v_j \)).

Construa um Meta-Hipergrafo \( U' = (V', H', E_H', \emptyset, M') \) sem relações, onde:

\begin{itemize}
    \item \( V' = V \cup V_d \), com \( V_d \) é inserido duplicata de vértices a partir do ponto de ramificação \( s \) e subgrafos além (incluindo duplicatas de \( v_j^d \), \( t^d \)) até chegar no vértice convergente \( t \).
    \item \( E_H' \) inclui Meta-Hiperarestas:

    \subitem Meta-Hiperarestas originais de \( E_H \) até a ramificação.
    \subitem Para ramo com \( v_i \): Meta-Hiperarestas para \( v_j \), \( t \), e subgrafo além.
    \subitem Para ramo sem \( v_i \) (via \( v_x \)): Meta-Hiperarestas para duplicatas \( V_d \) de forma similar às vértices originárias conectadas com outras \( v^d \) e \( v_x \), mas com corte abrupto (sem Meta-Hiperarestas além do correspondente a \( v_j^d \), representando proibição estrutural).

    \item \( \forall v \in V: |\{ i \mid 0 \leq i \leq j, v_i = v \}| \leq M(v) \).
    \item \( M'(v) = M(v) \) para \( v \in V \), \( M'(v^d) = M(v) \) para \( v^d \in V_d \).
\end{itemize}


\textbf{Existe tal Meta-Hipergrafo \( U' \) sem relações cujo conjunto de Meta-Hipercaminhos válidos \( P'_H \) coincide com o \( P_V \) e \( P_H \) de \( U \) via mapeamento bijetivo.}
\\
\\
Exemplo: \( P'_H = \delta(P_v) \) e \( P'_H = \delta(\pi(P_H)) \) dado \( \pi \) uma função que seleciona elemento em \( V \) (Elemento Original) dentro de \( w \).
\end{theorem}

\begin{proof}
A construção de \( U' \) utiliza duplicatas nos Meta-Hipervértices via \( \delta \) para codificar "modos" de ativação: o modo original (\( v \)) para caminhos que satisfazem a relação \( \implies \) (i.e., passando por \( v_i \)), e o modo duplicado (\( v^d \)) para caminhos que tentam violar a relação (sem \( v_i \)). Os Meta-Hipervértices compostos \( \delta(v) = \{v, v^d\} \) permitem escolha implícita no caminho induzido \( P_V' \), mas as Meta-Hiperarestas \( E_H' \) são definidas de forma a cortar progressão no modo duplicado após \( v_j^d \), replicando a restrição lógica sem \( R \).

Defina o mapeamento bijetivo \( \phi: P_V(U) \to P_H'(U') \) (e similarmente para \( P_H(U) \), pois \( H \) são singletons, \( P_H(U) \equiv P_V(U) \) via \( \pi(w) = v \in w \)) como \( \phi(P_V) = (\delta(v_0), \delta(v_1), \dots, \delta(v_k)) \), onde para caminhos válidos em \( U \) (que passam por \( v_i \) para ativar \( v_j \) e além), a escolha no caminho induzido em \( U' \) usa o modo original \( v \); para tentativas inválidas, o modo \( v^d \) é forçado pelo ramo via \( v_x \), mas cortado.

\textbf{Passo 1: Todo \( P_V \) válido em \( U \) mapeia para \( P_H' \) válido em \( U' \).}

Seja \( P_V = (v_0 = s, \dots, v_k) \) válido em \( U \). Como válido, se ativa \( v_j \) ou além, deve ter passado por \( v_i \), satisfazendo \( \implies \) em prefixos. O ramo é via \( v_i \), então em \( U' \), as Meta-Hiperarestas preservam conexões originais: \( E_H' \) inclui arestas de \( E_H \) para o modo original. Assim, \( P_H' = \phi(P_V) = (\delta(v_0), \dots, \delta(v_k)) \) satisfaz as condições de Meta-Hipercaminho em \( E_H' \) (arestas originais conectam subconjuntos nos modos \( v \)), e as visitas respeitam \( M' \) (idêntico a \( M \)). Como não usa modos \( v^d \) (forçados apenas no ramo sem \( v_i \)), não há corte, e \( P_H' \) é válido.

\textbf{Passo 2: Todo \( P_H' \) válido em \( U' \) mapeia para \( P_V \) válido em \( U \).}

Seja \( P_H' = (w_0, \dots, w_k) \) válido em \( U' \). Como \( R = \emptyset \), validade depende só de \( E_H' \) e \( M' \). Defina o inverso \( \phi^{-1}(P_H') = (\pi'(w_0), \dots, \pi'(w_k)) \), onde \( \pi'(w) = v \) se \( w = \{v\} \) ou o componente original \( v \in w = \{v, v^d\} \) (colapsando duplicatas para originais apenas se o caminho induzido usa modo válido). Para \( P_H' \) válido que alcança além de \( v_j \) (e.g., subgrafos dependentes), deve usar o modo original nos Meta-Hipervértices \( \delta(v) \), pois o modo \( v^d \) é cortado em \( E_H' \) após \( v_j^d \) (sem arestas além). Caminhos que tentam modo \( v^d \) (via ramo \( v_x \)) param em \( v_j^d \), não alcançando o final. Assim, \( P_H' \) válidos completos correspondem a caminhos usando modo \( v \), que mapeiam para \( P_V \) em \( U \) via colapso, e como usam ramo \( v_i \), satisfazem \( \implies \) e outras condições (edges e \( M \) preservados).
\newpage

\textbf{Passo 3: O mapeamento é bijetivo.}

\( \phi \) é injetivo: caminhos distintos em \( U \) levantam para Meta-Hipercaminhos distintos em \( U' \) (modos originais preservam estrutura). Surjetivo sobre válidos: todo \( P_H' \) válido em \( U' \) usa modo original (como duplicados cortam), colapsando bijetivamente para \( P_V \) válido em \( U \). Exemplo: \( P'_H = \delta(P_V) \) preserva validade, e \( \phi^{-1} = \pi \circ \delta^{-1} \) (selecionando original).

Portanto, os conjuntos coincidem via o mapeamento.
\end{proof}
\bigskip

TODO: Eu ainda tenho que verificar se todas as provas estão corretas e os teoremas e definições.

\end{document}
