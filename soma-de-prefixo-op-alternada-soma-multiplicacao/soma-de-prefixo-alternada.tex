\documentclass{article}
\usepackage{amsmath}
\usepackage{amssymb}
\usepackage{geometry}
\usepackage{hyperref}
\title{Algoritmo de Prefixo para Soma e Multiplicações Alternadas}
\author{}
\date{}
\begin{document}
\maketitle
\section{Introdução e Alguns Resultados}
Dado uma sequencia ordenada $A_n$, nós definimos uma função alternada $op(A)$ que segue:
\begin{equation}
op([a_0,a_1,a_2,a_3,a_4,a_5...])= (((((a_0+a_1)\cdot a_2)+a_3)\cdot a_4)+a_5)\cdot ...
\end{equation}
Dado $|A|=k$ podemos escrever a equação utilizando a propriedade distributiva de forma que:
\begin{equation}
op([a_0,a_1,a_2,a_3,a_4,a_5...])= 1(a_0 \cdot a_2 \cdot a_4 \cdot a_6 \dots) + a_1 (a_2 \cdot a_4 \cdot a_6 \dots) + a_3 (a_4 \cdot a_6 \dots) + a_ 5 (a_6 \dots) ...
\end{equation}
Utilizando notação de produtório:
\begin{equation}
op([a_0,a_1,a_2,a_3,a_4,a_5...])= 1 \prod_{i=0}^{\lfloor k/2 \rfloor} a_{2i} + a_1 \prod_{i=1}^{\lfloor k/2 \rfloor } a_{2i} + a_3 \prod_{i=2}^{\lfloor k/2 \rfloor} a_{2i} + a_ 5 \prod_{i=3}^{\lfloor k/2 \rfloor} a_{2i} ...
\end{equation}
E de forma compacta:
\begin{equation}
op([a_0,a_1,a_2,a_3,a_4,a_5...])=\prod_{j=0}^{\lfloor k/2 \rfloor } a_{2j} + \sum_{i=1}^{\lfloor (k+1)/2 \rfloor} ( a_{2i-1} \prod_{j=i}^{\lfloor k/2 \rfloor} a_{2j} )
\end{equation}
Suponhamos que exista um $S$ tal que $A \subset S$ e $s_{-1} = 1$ e $s_n=a_n$, logo podemos expressar a formula da seguinte maneira:
\begin{equation}
op(A)=\sum_{i=0}^{\lfloor (k+1)/2 \rfloor} ( s_{2i-1} \prod_{j=i}^{\lfloor k/2 \rfloor} s_{2j} )
\end{equation}
Aonde $s \in S$.
para uma sub-sequencia ordenada $A_{[\alpha, \beta]}$ em um intervalo de indices, aonde $A_{[\alpha, \beta]} \subset A$, $A_{[\alpha, \beta]}= [a_\alpha,a_{\alpha+1},a_{\alpha+2},a_{\alpha+3},a_{\alpha+4},a_{\alpha+4}... a_{\beta}]$ e $|B|= s = k - \beta < k $, temos para $op(A_{[\alpha, \beta]})$:
\begin{equation}
op(A_{[\alpha, \beta]})= \prod_{j=0}^{\lfloor s/2 \rfloor } a_{2j +\alpha} + \sum_{i=1}^{\lfloor (s+1)/2 \rfloor} ( a_{2i-1 +\alpha} \prod_{j=i}^{\lfloor s/2 \rfloor} a_{2j +\alpha} )
\end{equation}
ou também podemos escrever, para $\alpha$ par:
\begin{equation}
op(A_{[\alpha, \beta]})=  a_{\alpha} \prod_{i=\lfloor \alpha/2 \rfloor}^{\lfloor\beta/2\rfloor} a_{2i} + a_{\alpha+1} \prod_{i=\lfloor \alpha/2 \rfloor +1}^{\lfloor\beta/2\rfloor} a_{2i} + a_{\alpha+3} \prod_{i=\lfloor \alpha/2 \rfloor+2}^{\lfloor\beta/2\rfloor } a_{2i} + \dots + a_\beta \prod_{i=\lfloor \beta/2 \rfloor}^{\lfloor\beta/2\rfloor}
a_{2i}
\end{equation}
Aonde é escrito de forma extensa e é feita uma mudança de indices, e também pois:
\begin{equation}
op([a_{\alpha},a_{\alpha+1},a_{\alpha+2},...,a_{\beta}])= a_{\alpha}(a_{\alpha+2}  \cdot a_{\alpha+4} \cdot a_{\alpha+6}  \cdot \dots \cdot a_\beta) + a_{\alpha+1} (a_{\alpha+4} \cdot a_{\alpha+6}  \cdot \dots \cdot a_\beta) + a_{\alpha+3} (a_{\alpha+6}  \cdot \dots a_\beta) + a_{\alpha+5} (\dots a_\beta) ...
\end{equation}
e para $\alpha$ impar:
\begin{equation}
op(A_{[\alpha, \beta]})=  a_{\alpha} \prod_{i=\lfloor\alpha/2\rfloor}^{\lfloor\beta/2\rfloor} a_{2i-1} + a_{\alpha+2} \prod_{i=\lfloor \alpha/2 \rfloor +1}^{\lfloor\beta/2\rfloor} a_{2i-1} + a_{\alpha+4} \prod_{i=\lfloor \alpha/2 \rfloor+2}^{\lfloor\beta/2\rfloor } a_{2i-1} + \dots + a_\beta \prod_{i=\lfloor \beta/2 \rfloor}^{\lfloor\beta/2\rfloor} a_{2i-1}
\end{equation}
pois, respectivamente:
\begin{equation}
op([a_{\alpha},a_{\alpha+1},a_{\alpha+2},...,a_{\beta}])= a_{\alpha}(a_{\alpha+1}  \cdot a_{\alpha+3} \cdot a_{\alpha+5}  \cdot \dots \cdot a_\beta) + a_{\alpha+2} (a_{\alpha+3} \cdot a_{\alpha+5}  \cdot \dots \cdot a_\beta) + a_{\alpha+4} (a_{\alpha+5}  \cdot \dots a_\beta) + a_{\alpha+6} (\dots a_\beta) ...
\end{equation}
Repare que para a subsequencia dependendendo do valor de $\alpha$ as paridades mudam de forma que para $\alpha$ impar temos $2j+\alpha$ impar e $2i-1+\alpha$ par, e se $\alpha$ é par temos $2j+\alpha$ par e $2i-1+\alpha$ impar.
Vamos definir a sub-sequencias $A_{[0,k-1]}$ que não ultimo indice de forma que:
$A_{[0,k-1]} = [a_0,a_1,a_2,.., a_{k-1}]$ temos que:
\begin{equation}
op(A_{[0,k-1]})= \begin{cases}
op(A) - a_k \text{ e } k \text{ é impar } \
\frac{op(A)}{a_k} \text{ e } k \text{ é par } \
\end{cases}
\end{equation}
\subsection{Prova 1}
Temos duas possibilidades, para k impar:
\begin{equation}
op([a_0,a_1,a_2,a_3,a_4,a_5... a_{k-1]})= (((((a_0+a_1)\cdot a_2)+a_3)\cdot a_4)+a_5)\cdot ... + a_{k}
\end{equation}
e para k par:
\begin{equation}
op([a_0,a_1,a_2,a_3,a_4,a_5... a_{k-1]})= ((((((a_0+a_1)\cdot a_2)+a_3)\cdot a_4)+a_5)\cdot ...) a_{k}
\end{equation}
sabemos $op(A_{[0,k-1]})$ é op(A) sem ultimo elemento, logo:
\begin{equation}
op(A_{[0,k-1]})= \begin{cases}
op(A) - a_k \text{ e } k \text{ é impar } \
\frac{op(A)}{a_k} \text{ e } k \text{ é par } \
\end{cases}
\end{equation}
Q.E.D

\section{Algoritmo}

Dado uma sequencia ordenada $A_n$, nós definimos uma função alternada $op(A)$ que segue:

\begin{equation*}
op([a_0,a_1,a_2,a_3,a_4,a_5...])= (((((a_0+a_1)\cdot a_2)+a_3)\cdot a_4)+a_5)\cdot ...
\end{equation*}

Definimos as seguintes sequencias de multiplicações cumulativas:

\begin{equation*}
M^{+} = [ a_0, a_0 \cdot a_2, a_0 \cdot a_2 \cdot a_4, a_0 \cdot a_2 \cdot a_4 \cdot a_6, \dots]
\end{equation*}

\begin{equation*}
M^{-} = [ a_1, a_1 \cdot a_3, a_1 \cdot a_3 \cdot a_5, a_1 \cdot a_3 \cdot a_5 \cdot a_7, \dots]
\end{equation*}

Com isso conseguimos montar duas expressões $m_{+}$ e $m_{-}$ que utilizam indice inicial $a$ e indice final $b$ para pegar o intervalo em multiplicação de prefixos:

\begin{equation*}
\delta_{+}(a, b) = \frac{M_{b}^{+}}{M_{a}^{+}}
\end{equation*}

\begin{equation*}
\delta_{-}(a, b) = \frac{M_{b}^{-}}{M_{a}^{-}}
\end{equation*}

Definimos assim uma função para o intervalo $(\alpha, \beta)$:

\begin{equation*}
\Delta = \begin{cases}
\delta_{-}((\alpha - 1)/2, (\beta - 1)/2) \text{ e } \alpha \text{ é impar } \\
\delta_{+}(\alpha/2, \beta/2) \text{ e } \alpha \text{ é par } \\
\end{cases}
\end{equation*}

Também definimos que são sequencias cumulativas de somas e multiplicações alternadas:

\begin{equation*}
P^{+} = [ op(A_{[0,0]}), op(A_{[0,1]}), op(A_{[0,2]}), op(A_{[0,3]}) \dots op(A_{[0,k]})]
\end{equation*}

\begin{equation*}
P^{-} = [ op(A_{[1,0]}), op(A_{[1,1]}), op(A_{[1,2]}), op(A_{[1,3]}) \dots op(A_{[1,k]})]
\end{equation*}

Aonde $|A|=k$

Temos a seguinte formula como válida para todo $\alpha < \beta < k$ e $\alpha, \beta \in \mathbb{N}$:

\begin{equation*}
op(A_{[\alpha, \beta]}) = \begin{cases}
P^{-}_{\beta-1} + \Delta \cdot (a_\alpha - P^{-}_{\alpha-1}) & \text{se } \alpha \text{ é ímpar} \\
P^{+}_{\beta} + \Delta \cdot (a_\alpha - P^{+}_{\alpha}) & \text{se } \alpha \text{ é par}
\end{cases}
\end{equation*}

\section{Prova do Algoritmo}

Dado a seguinte formula, devemos provar que equação é válida:

\begin{equation*}
op(A_{[\alpha, \beta]}) = \begin{cases}
P^{-}_{\beta-1} + \Delta \cdot (a_\alpha - P^{-}_{\alpha-1}) & \text{se } \alpha \text{ é ímpar} \\
P^{+}_{\beta} + \Delta \cdot (a_\alpha - P^{+}_{\alpha}) & \text{se } \alpha \text{ é par}
\end{cases}
\end{equation*}

Temos as seguintes sequências:

\begin{equation*}
P^{+} = [ op(A_{[0,0]}), op(A_{[0,1]}), op(A_{[0,2]}), op(A_{[0,3]}) \dots op(A_{[0,k]})]
\end{equation*}

\begin{equation*}
P^{-} = [ op(A_{[1,0]}), op(A_{[1,1]}), op(A_{[1,2]}), op(A_{[1,3]}) \dots op(A_{[1,k]})]
\end{equation*}

Podemos fazer a simples substituição:

\begin{equation*}
op(A_{[\alpha, \beta]}) = \begin{cases}
op(A_{[1,\beta-1]}) + \Delta \cdot (a_\alpha - op(A_{[1,\alpha-1]})) & \text{se } \alpha \text{ é ímpar} \\
op(A_{[0,\beta]}) + \Delta \cdot (a_\alpha - op(A_{[0,\alpha]})) & \text{se } \alpha \text{ é par}
\end{cases}
\end{equation*}

É trivial repararmos que para todo índice $m$ o elemento da matriz de prefixos de multiplicação é:

\begin{equation*}
M^{+}_{m} = \prod_{i=0}^{m}{a_{2i}}
\end{equation*}

\begin{equation*}
M^{-}_{m} = \prod_{i=0}^{m}{a_{2i-1}}
\end{equation*}

E temos:

\begin{equation*}
\delta_{+}(a, b) = \frac{M_{b}^{+}}{M_{a}^{+}} = \frac{\prod_{i=0}^{b} a_{2i}}{\prod_{i=0}^{a} a_{2i}} = \prod_{i=a}^{b} a_{2i}
\end{equation*}

\begin{equation*}
\delta_{-}(a, b) = \frac{M_{b}^{-}}{M_{a}^{-}} = \frac{\prod_{i=0}^{b} a_{2i-1}}{\prod_{i=0}^{a} a_{2i-1}} = \prod_{i=a}^{b} a_{2i-1}
\end{equation*}

Analisamos essa expressão:

\begin{equation*}
\Delta = \begin{cases}
\delta_{-}((\alpha-1)/2, (\beta-1)/2) \text{ e } \alpha \text{ é impar } \\
\delta_{+}(\alpha/2, \beta/2) \text{ e } \alpha \text{ é par } \\
\end{cases}
\end{equation*}

Substituímos:

\begin{equation*}
\Delta = \begin{cases}
 \prod_{i=(\alpha-1)/2}^{(\beta-1)/2} a_{2i-1} \text{ e } \alpha \text{ é impar } \\
\prod_{i=\alpha/2}^{\beta/2} a_{2i}  \text{ e } \alpha \text{ é par } \\
\end{cases}
\end{equation*}

É possível também fazer uso da função chão de forma que, e assim podendo igualar os índices das somatórias:

\begin{equation*}
\Delta = \begin{cases}
 \prod_{i=\lfloor\alpha/2\rfloor}^{\lfloor\beta/2\rfloor} a_{2i-1} \text{ e } \alpha \text{ é impar } \\
\prod_{i=\lfloor\alpha/2\rfloor}^{\lfloor\beta/2\rfloor} a_{2i}  \text{ e } \alpha \text{ é par } \\
\end{cases}
\end{equation*}

Fazemos também substituição na expressão original:

\begin{equation*}
op(A_{[\alpha, \beta]}) = \begin{cases}
op(A_{[1,\beta-1]}) + \prod_{i=\lfloor\alpha/2\rfloor}^{\lfloor\beta/2\rfloor} a_{2i-1} \cdot (a_\alpha - op(A_{[1,\alpha-1]})) & \text{se } \alpha \text{ é ímpar} \\
op(A_{[0,\beta]}) + \prod_{i=\lfloor\alpha/2\rfloor}^{\lfloor\beta/2\rfloor} a_{2i} \cdot (a_\alpha - op(A_{[0,\alpha]})) & \text{se } \alpha \text{ é par}
\end{cases}
\end{equation*}

Vamos tentar provar essa expressão dividindo em casos.

\section*{Caso 1 - $\alpha$ sendo par}

Vamos analisar a expressão:

\begin{equation*}
op(A_{[0,\beta]}) + \prod_{i=\lfloor\alpha/2\rfloor}^{\lfloor\beta/2\rfloor} a_{2i} \cdot (a_\alpha - op(A_{[0,\alpha]})) 
\end{equation*}

Expandindo:

\begin{equation*}
op(A_{[0,\beta]}) + a_\alpha\prod_{i=\lfloor\alpha/2\rfloor}^{\lfloor\beta/2\rfloor} a_{2i} - op(A_{[0,\alpha]})\prod_{i=\lfloor\alpha/2\rfloor}^{\lfloor\beta/2\rfloor} a_{2i} 
\end{equation*}

Temos:

\begin{equation*}
op(A_{[0,\beta]}) = 1 \prod_{i=0}^{\lfloor \beta/2 \rfloor} a_{2i} + a_1 \prod_{i=1}^{\lfloor \beta/2 \rfloor } a_{2i} + a_3 \prod_{i=2}^{\lfloor \beta/2 \rfloor} a_{2i} + a_ 5 \prod_{i=3}^{\lfloor \beta/2 \rfloor} a_{2i} + \dots + a_\beta \prod_{i=\lfloor \beta/2 \rfloor}^{\lfloor \beta/2 \rfloor} a_{2i}
\end{equation*}

\begin{equation*}
op(A_{[0,\alpha]}) = 1 \prod_{i=0}^{\lfloor \alpha/2 \rfloor} a_{2i} + a_1 \prod_{i=1}^{\lfloor \alpha/2 \rfloor } a_{2i} + a_3 \prod_{i=2}^{\lfloor \alpha/2 \rfloor} a_{2i} + a_ 5 \prod_{i=3}^{\lfloor \alpha/2 \rfloor} a_{2i} + \dots + a_\alpha \prod_{i=\lfloor \alpha/2 \rfloor}^{\lfloor \alpha/2 \rfloor} a_{2i}
\end{equation*}

Também:

\begin{equation*}
op(A_{[0,\alpha]})\prod_{i=\lfloor\alpha/2\rfloor}^{\lfloor\beta/2\rfloor} a_{2i} = \prod_{i=\lfloor\alpha/2\rfloor}^{\lfloor\beta/2\rfloor} a_{2i} (1 \prod_{i=0}^{\lfloor \alpha/2 \rfloor} a_{2i} + a_1 \prod_{i=1}^{\lfloor \alpha/2 \rfloor } a_{2i} + a_3 \prod_{i=2}^{\lfloor \alpha/2 \rfloor} a_{2i} + a_ 5 \prod_{i=3}^{\lfloor \alpha/2 \rfloor} a_{2i} + \dots + a_\alpha \prod_{i=\lfloor \alpha/2 \rfloor}^{\lfloor \alpha/2 \rfloor} a_{2i})
\end{equation*}

Podemos mudar os índices de todos os multiplicatórios:

\begin{equation*}
op(A_{[0,\alpha]})\prod_{i=\lfloor\alpha/2\rfloor}^{\lfloor\beta/2\rfloor} a_{2i} = 1 \prod_{i=0}^{\lfloor\beta/2\rfloor} a_{2i} + a_1 \prod_{i=1}^{\lfloor\beta/2\rfloor } a_{2i} + a_3 \prod_{i=2}^{\lfloor\beta/2\rfloor} a_{2i} + a_ 5 \prod_{i=3}^{\lfloor\beta/2\rfloor} a_{2i} + \dots + a_\alpha \prod_{i=\lfloor \alpha/2 \rfloor}^{\lfloor\beta/2\rfloor} a_{2i}
\end{equation*}

Com isso fica claro que:

\begin{equation*}
op(A_{[0,\beta]}) - op(A_{[0,\alpha]})\prod_{i=\lfloor\alpha/2\rfloor}^{\lfloor\beta/2\rfloor} a_{2i} = a_{\alpha+1} \prod_{i=\lfloor \alpha/2 \rfloor}^{\lfloor\beta/2\rfloor} a_{2i} + a_{\alpha+3} \prod_{i=\lfloor \alpha/2 \rfloor+1}^{\lfloor\beta/2\rfloor } a_{2i} + \dots + a_\beta \prod_{i=\lfloor \beta/2 \rfloor}^{\lfloor\beta/2\rfloor} a_{2i}
\end{equation*}

Somando $ a_\alpha\prod_{i=\lfloor\alpha/2\rfloor}^{\lfloor\beta/2\rfloor}$, nós temos:

\begin{equation*}
op(A_{[0,\beta]}) - op(A_{[0,\alpha]})\prod_{i=\lfloor\alpha/2\rfloor}^{\lfloor\beta/2\rfloor} a_{2i} + a_\alpha\prod_{i=\lfloor\alpha/2\rfloor}^{\lfloor\beta/2\rfloor} =a_\alpha\prod_{i=\lfloor\alpha/2\rfloor}^{\lfloor\beta/2\rfloor} +  a_{\alpha+1} \prod_{i=\lfloor \alpha/2 \rfloor}^{\lfloor\beta/2\rfloor} a_{2i} + a_{\alpha+3} \prod_{i=\lfloor \alpha/2 \rfloor+1}^{\lfloor\beta/2\rfloor } a_{2i} + \dots + a_\beta \prod_{i=\lfloor \beta/2 \rfloor}^{\lfloor\beta/2\rfloor}
\end{equation*}

Que é precisamente a definição de $op(A_{[\alpha, \beta]})$ para $\alpha$ par, então:

\begin{equation*}
op(A_{[\alpha, \beta]}) = op(A_{[0,\beta]}) - op(A_{[0,\alpha]})\prod_{i=\lfloor\alpha/2\rfloor}^{\lfloor\beta/2\rfloor} a_{2i} + a_\alpha\prod_{i=\lfloor\alpha/2\rfloor}^{\lfloor\beta/2\rfloor} a_{2i}
\end{equation*}

Ou seja:

\begin{equation*}
op(A_{[\alpha, \beta]}) = op(A_{[0,\beta]}) + a_\alpha\prod_{i=\lfloor\alpha/2\rfloor}^{\lfloor\beta/2\rfloor} a_{2i} - op(A_{[0,\alpha]})\prod_{i=\lfloor\alpha/2\rfloor}^{\lfloor\beta/2\rfloor} a_{2i} 
\end{equation*}

Com isso provamos para $\alpha$ par a expressão equivale a $op(A_{[\alpha, \beta]})$.

\section*{Caso 2 - $\alpha$ sendo ímpar}
Vamos analisar a expressão:
\begin{equation*}
op(A_{[1,\beta-1]}) + \prod_{i=\lfloor\alpha/2\rfloor}^{\lfloor\beta/2\rfloor} a_{2i-1} \cdot (a_\alpha - op(A_{[1,\alpha-1]}))
\end{equation*}
expandindo:
\begin{equation*}
op(A_{[1,\beta-1]}) + a_\alpha\prod_{i=\lfloor\alpha/2\rfloor}^{\lfloor\beta/2\rfloor} a_{2i-1} - op(A_{[1,\alpha-1]})\prod_{i=\lfloor\alpha/2\rfloor}^{\lfloor\beta/2\rfloor} a_{2i-1}
\end{equation*}
temos:
\begin{equation*}
op(A_{[1,\beta-1]}) = a_0 \prod_{i=0}^{\lfloor \beta/2 \rfloor} a_{2i-1} + a_2 \prod_{i=1}^{\lfloor \beta/2 \rfloor } a_{2i-1} + a_4 \prod_{i=2}^{\lfloor \beta/2 \rfloor} a_{2i-1} + a_6 \prod_{i=3}^{\lfloor \beta/2 \rfloor} a_{2i-1} + \dots + a_\beta \prod_{i=\lfloor \beta/2 \rfloor}^{\lfloor \beta/2 \rfloor} a_{2i-1}
\end{equation*}
\begin{equation*}
op(A_{[1,\alpha-1]}) = a_0 \prod_{i=0}^{\lfloor \alpha/2 \rfloor} a_{2i-1} + a_2 \prod_{i=1}^{\lfloor \alpha/2 \rfloor } a_{2i-1} + a_4 \prod_{i=2}^{\lfloor \alpha/2 \rfloor} a_{2i-1} + a_6 \prod_{i=3}^{\lfloor \alpha/2 \rfloor} a_{2i-1} + \dots + a_\alpha \prod_{i=\lfloor \alpha/2 \rfloor}^{\lfloor \alpha/2 \rfloor} a_{2i-1}
\end{equation*}
também:
\begin{equation*}
op(A_{[1,\alpha-1]})\prod_{i=\lfloor\alpha/2\rfloor}^{\lfloor\beta/2\rfloor} a_{2i-1} = \prod_{i=\lfloor\alpha/2\rfloor}^{\lfloor\beta/2\rfloor} a_{2i} (a_0 \prod_{i=0}^{\lfloor \alpha/2 \rfloor} a_{2i-1} + a_2 \prod_{i=1}^{\lfloor \alpha/2 \rfloor } a_{2i-1} + a_4 \prod_{i=2}^{\lfloor \alpha/2 \rfloor} a_{2i-1} + a_6 \prod_{i=3}^{\lfloor \alpha/2 \rfloor} a_{2i-1} + \dots + a_\alpha \prod_{i=\lfloor \alpha/2 \rfloor}^{\lfloor \alpha/2 \rfloor} a_{2i-1})
\end{equation*}
podemos mudar os indices de todos os multiplicatórios:
\begin{equation*}
op(A_{[1,\alpha-1]})\prod_{i=\lfloor\alpha/2\rfloor}^{\lfloor\beta/2\rfloor} a_{2i-1} = a_0 \prod_{i=0}^{\lfloor \beta/2 \rfloor} a_{2i-1} + a_2 \prod_{i=1}^{\lfloor \beta/2 \rfloor } a_{2i-1} + a_4 \prod_{i=2}^{\lfloor \beta/2 \rfloor} a_{2i-1} + a_6 \prod_{i=3}^{\lfloor \beta/2 \rfloor} a_{2i-1} + \dots + a_\alpha \prod_{i=\lfloor \alpha/2 \rfloor}^{\lfloor \beta/2 \rfloor} a_{2i-1}
\end{equation*}
com isso fica claro que:
\begin{equation*}
op(A_{[1,\beta-1]}) - op(A_{[1,\alpha-1]})\prod_{i=\lfloor\alpha/2\rfloor}^{\lfloor\beta/2\rfloor} a_{2i-1} = a_{\alpha+2} \prod_{i=\lfloor \alpha/2 \rfloor}^{\lfloor\beta/2\rfloor} a_{2i-1} + a_{\alpha+4} \prod_{i=\lfloor \alpha/2 \rfloor+1}^{\lfloor\beta/2\rfloor } a_{2i-1} + \dots + a_\beta \prod_{i=\lfloor \beta/2 \rfloor}^{\lfloor\beta/2\rfloor} a_{2i-1}
\end{equation*}
somamos $a_\alpha\prod_{i=\lfloor\alpha/2\rfloor}^{\lfloor\beta/2\rfloor}$ pois:
\begin{equation*}
op(A_{[1,\beta-1]}) - op(A_{[1,\alpha-1]})\prod_{i=\lfloor\alpha/2\rfloor}^{\lfloor\beta/2\rfloor} a_{2i-1} + a_\alpha\prod_{i=\lfloor\alpha/2\rfloor}^{\lfloor\beta/2\rfloor} a_{2i-1} = a_\alpha\prod_{i=\lfloor\alpha/2\rfloor}^{\lfloor\beta/2\rfloor}a_{2i-1} + a_{\alpha+2} \prod_{i=\lfloor \alpha/2 \rfloor}^{\lfloor\beta/2\rfloor} a_{2i-1} + a_{\alpha+4} \prod_{i=\lfloor \alpha/2 \rfloor+1}^{\lfloor\beta/2\rfloor } a_{2i-1} + \dots + a_\beta \prod_{i=\lfloor \beta/2 \rfloor}^{\lfloor\beta/2\rfloor} a_{2i-1}
\end{equation*}
Que é precisamente a definição de $op(A_{[\alpha, \beta]})$ para $\alpha$ impar, então:
\begin{equation*}
op(A_{[\alpha, \beta]}) = op(A_{[1,\beta-1]}) + \prod_{i=\lfloor\alpha/2\rfloor}^{\lfloor\beta/2\rfloor} a_{2i-1} \cdot (a_\alpha - op(A_{[1,\alpha-1]}))
\end{equation*}
Com isso provamos para $\alpha$ impar a expressão equivale a $op(A_{[\alpha, \beta]})$.

\section{Finalizando Prova}

Como provamos que a expressão:

\begin{equation*}
op(A_{[\alpha, \beta]}) = \begin{cases}
op(A_{[1,\beta-1]}) + \prod_{i=\lfloor\alpha/2\rfloor}^{\lfloor\beta/2\rfloor} a_{2i-1} \cdot (a_\alpha - op(A_{[1,\alpha-1]})) & \text{se } \alpha \text{ é ímpar} \\
op(A_{[0,\beta]}) + \prod_{i=\lfloor\alpha/2\rfloor}^{\lfloor\beta/2\rfloor} a_{2i} \cdot (a_\alpha - op(A_{[0,\alpha]})) & \text{se } \alpha \text{ é par}
\end{cases}
\end{equation*}

é uma expressão válida, e temos que essa expressão e definida através de simples substituições que são reversíveis da sequencias da nossa expressão original, que é: 

\begin{equation*}
op(A_{[\alpha, \beta]}) = \begin{cases}
P^{-}_{\beta-1} + \Delta \cdot (a_\alpha - P^{-}_{\alpha-1}) & \text{se } \alpha \text{ é ímpar} \\
P^{+}_{\beta} + \Delta \cdot (a_\alpha - P^{+}_{\alpha}) & \text{se } \alpha \text{ é par}
\end{cases}
\end{equation*}

Por tanto a seguinte equação é válida.




\section{Implementação em Python}
\begin{verbatim}
def alternated_sum_mul(arr):
if len(arr) == 0: return 0
result, seq = arr[0], [arr[0]]
for i in range(1, len(arr)):
if i % 2 == 1:
result += arr[i]
else:
result *= arr[i]
seq.append(result)
return seq
def prefix_mult_parity(arr, par=1):
result, prefix = 1, []
for i in range(1 if par else 0, len(arr), 2):
result *= arr[i]
prefix.append(result)
return prefix
def all_ranges_ordered_op(arr):
prefix_sm_0, prefix_sm_1 = alternated_sum_mul(arr), alternated_sum_mul(arr[1:])
prefix_m_0, prefix_m_1  = prefix_mult_parity(arr, 0), prefix_mult_parity(arr, 1)
r = []
comb = [(i, j) for i in range(len(arr)) for j in range(i + 1, len(arr))]
for i, j in comb:
if (i&1):
jm, im = j-1, i-1
delta = prefix_m_1[jm//2]//prefix_m_1[im//2]
r.append(prefix_sm_1[jm]+delta*(arr[i]-prefix_sm_1[im]))
else:
delta = prefix_m_0[j//2]//prefix_m_0[i//2]
r.append(prefix_sm_0[j]+delta*(arr[i]-prefix_sm_0[i]))
return sorted(dict.fromkeys(r + arr))
\end{verbatim}
\end{document}