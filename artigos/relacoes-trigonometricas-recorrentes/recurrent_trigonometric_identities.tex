\documentclass{article}
\usepackage[OT6,T1]{fontenc}
\usepackage[utf8]{inputenc}
\usepackage{amssymb} 
\usepackage{amsthm}
\usepackage{amsmath}
\usepackage{amsfonts}
\newtheorem{teorema}{Teorema}
\newtheorem{lema}[teorema]{Lema}
\newtheorem{conjectura}{Conjectura}
\newtheorem{proposicao}{Proposição}
\newtheorem{corolario}{Corolário}
\renewenvironment{proof}[1][Prova]{\par\noindent\textbf{#1.} \rmfamily}{\hfill$\square$\par}
\newcommand{\armenian}{\fontencoding{OT6}\fontfamily{cmr}\selectfont}
\DeclareTextFontCommand{\textarmenian}{\armenian}
\author{Anjo}
\begin{document}
\title{Funções Trigonométricas Compostas e Recursivas}
\sloppy
\maketitle

\section{Introdução}
Quando eu era jovem eu gostava muito de explorar
funções mátematicas e entre uma das minhas explorações mais comuns estavam
com o fato de eu brincar com as funçoes trigonométricas mais especificamente
composições do tipo $\cos(\cos(x))$ e $\arcsin(\cos(x))$, meu interesse na criação desse documento se 
iniciou com amigo Hobbista Mátematico Novato chamado Kaio, que criou determina
qual hora o sol nasce ou se põe em determinada latitude e longitude do planeta.
Esse programa que Kaio criou em suas exploraçoes pessoas faziam bastante uso funções trigonométricas compostas.

Vejo que é muito comum na parte de pesquisa matématica que matemáticos se concentrem na criação de teorias e resolução de problemas complexos
e acabem se esquecendo de resolver e catalogar problemas trigonométricos simples porém pouco explorados. Um dos poucos lugares que eu encontrei fala sobre esses tipos de composições mais incomuns de funções trigonométricas se
encontram em alguns exercicios do livro de Calculo de Spivak(exercicio 15-18) que fala justamente sobre $arcsin(cos(x))$, mesmo assim pouco se é falado e catalogado sobre elas
talvez, porque muito se ve que tais composiçoes não envolvem identidades trigonométricas simples e se crer que muitas dessas expressões não se simplificam sem retrições adicionais.

Porém é importante que exista estudo sobre elas, já que mais e mais matemáticos novatos irão se deparar com elas, mas não só por isso mas como também o fato que "olhar aonde não é olhado" 
pode trazer resultados relevantes ou até criação de novas teorias. 

Também caso for de interesse a utização dessas funções trigonométricas compostas em programas
como foi no caso do Kaio, eu posso dizer com garantia que é possivel criar programas performaticos simples sem necessáriamente calcular $arcsin(cos(x))$ 
diretamente, tornando tal conhecimento util para optimização dessas trigonométricas compostas em computadores.

\section{Notação}

Para que não tenha confusões com notação já conhecida $\cos^2(x)=(\cos(x))^2$ eu proponho a seguinte notação
para trigonométricas compostas, em que: 
$$\cos_2(x)=\cos(\cos(x))$$
também: 
$$\cos_n(x)=\cos \circ \cos \circ ... \circ \cos(x)$$
\break
A mesma notação também pode ser utilizada para demais funções trigonométricas.

\section{Resultados Operações Recursivas }

Dado $x \in \mathbb{R}$ recursão infinita abaixo.

$$ \lim\limits_{n \to \infty} \cos_n(x) =  0.739085.. $$
A constante $0.739085..$ se refere conhecido Número de Dottie denotado por $\textarmenian{ա} = 0.739085..$
quando aplicamos a função cosseno repetidademente tende a converger para ponto fixo, 
o ponto fixo pode ser encontrado como sendo  a unica solução real para equação $\cos(x)=x$, da mesma forma para o seno temos:

$$ \lim\limits_{n \to \infty} \sin_n(x) =  0 $$ 
Sendo também ponto fixo de seno, ou seja a solução para equação $\sin(x)=x$ dado 
$x \in \mathbb{R}$. 

\begin{conjectura}
  Dado uma sequencia infinita de funções recursivas $\{f_n\}_{n \in \mathbb{N}}$ aonde $f_0=f$ e $f_n=f_{n-1} \circ f$ e dado conjunto S temos $Y_n, X_n \subseteq S$ $f_n:X_n \to Y_n$ e um $x \in X_0$  aonde $x$ é ponto fixo do qual
  $f(x)=x$, para todo $y \in X_0$ aonde $\exists k \in \mathbb{N}:f_k(y)=x$  temos que:
$$\lim\limits_{n \to \infty} f_n(y) =  x$$  

\end{conjectura}

\begin{conjectura}
  Dado uma sequencia infinita de funções periódicas recursivas nos numéros reais  $\{f_n\}_{n \in \mathbb{N}}$ 
  aonde $f_0=f$ tal que existe um intervalo I aonde $I \subseteq \mathbb{R}$ que $\exists T \in I, \forall x \in \mathbb{R}: f(x)=f(x+T)$ e $f_n=f_{n-1} \circ f$ e $f_n$ é uma sobrejeção 
   e $\{i \in I : f(i)\} := \mathbb{R}$, para todo $x \in \mathbb{R}$  aonde $x$ é ponto fixo do qual $f(x)=x$ temos que:
  $$\exists k \in I, \forall m \in \mathbb{N}: \lim\limits_{n \to \infty} f_n(k+mT) =  x$$  

\end{conjectura}
\break

Para função tangente sendo aplicada recursivamente temos que a função
diverge, porém há valores em $x$ no quais a sequencia converge, dado $k \in \mathbb{N}$ os pontos no qual $\tan(x)=0$ são $x=k\pi$
como sabemos $\tan(0)=0$ a aplicação da função recursivamente naturalmente resultará em zero. Esses também são resultados para equação $\tan(x)=x$.

Logo temos para para todo $k \in \mathbb{N}$ temos:
$$\lim\limits_{n \to \infty} \tan_n(k\pi) =  0$$

Para função $\arctan$ já temos um ponto fixo de imediato que é $\arctan(0)=0$ e $\frac{d}{dx}(\arctan(x)) = \frac{1}{1 + x^2}$
o seguinte resultado pode ser provado através do Método de Newton aonde sequencia passa se aproximar de zero:
$$\lim\limits_{n \to \infty} \arctan_n(x) =  0$$

Para as funções $\arccos$ e $\arcsin$ certos detalhes devem ser reparados elas não são definida para todos números reais
e sim para intervalo $[-1,1]$, então a medida que elas são aplicadas recursivamente seu dominio vai diminuindo, convergendo para um dominio de um unico ponto.

$$ \text{Dom}(\lim\limits_{n \to \infty} \arccos_n):= \{\textarmenian{ա} \}$$
$$ \text{Dom}(\lim\limits_{n \to \infty} \arcsin_n):= \{0\}$$

\begin{conjectura}
  Dado uma sequencia infinita de funções recursivas $\{f_n\}_{n \in \mathbb{N}}$ aonde $f_0=f$ e $f_n=f_{n-1} \circ f$ e dado conjunto S temos $Y_n, X_n \subseteq S$ e $Y_0=X_0$ $f_n:X_n \cap Y_{n} \to Y_{n+1}$,
  temos que conjunto dominio e imagem de $f_n$ são filtros em S e:
$$\text{Dom}(f_n):=\bigcap_{k \leq n}{\text{Dom}(f_k)} $$
$$\text{Img}(f_n):=\bigcap_{k \leq n}{\text{Img}(f_k)} $$ 

\end{conjectura}

O dominio de $\arccos_n$ e $\arcsin_n$ são pontos fixos, mas deve ser reparado que imagem se mantém um intervalo fixo e a medida razão entre o dominio e imagem vai mudando,
por exemplo imagem de $\arccos_n$ é $[0, \pi]$ logo:


$$\{\forall n \in \mathbb{N} \land n > 1:x \in \text{Dom}(\arccos_n): \arccos_n(x)\} := [0, \frac{\pi}{2}]$$
o que implica que o limite do conjunto imagem:

$$\lim\limits_{n \to \infty}\{x \in \{\textarmenian{ա} \}: \arccos_n(x)\} := [0, \frac{\pi}{2}]$$

\break
\begin{corolario}
  $$\arccos_n: {\textarmenian{ա}} \to [0, \frac{\pi}{2}]$$
\end{corolario}


Nesse caso apesar que limite ordinário da função converge em determinado ponto, o limite do 
conjunto demonstra que resultado diverge.


\end{document}